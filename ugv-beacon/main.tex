\documentclass[journal,12pt,twocolumn]{IEEEtran}
\IEEEoverridecommandlockouts
\usepackage{setspace}
\usepackage{gensymb}
\singlespacing
\usepackage[cmex10]{amsmath}
\usepackage{amsthm}
\usepackage{mathrsfs}
\usepackage{txfonts}
\usepackage{stfloats}
\usepackage{bm}
\usepackage{cite}
\usepackage{cases}
\usepackage{subfig}
\usepackage{longtable}
\usepackage{multirow}
\usepackage{enumitem}
\usepackage{mathtools}
\usepackage{tikz}
\usepackage{circuitikz}
\usepackage{verbatim}
\usepackage[breaklinks=true]{hyperref}
\usepackage{tkz-euclide} % loads  TikZ and tkz-base
\usepackage{listings}
\usepackage{color}    
\usepackage{array}    
\usepackage{longtable}
\usepackage{calc}     
\usepackage{multirow} 
\usepackage{hhline}   
\usepackage{ifthen}   
\usepackage{lscape}     
\usepackage{chngcntr}
\usepackage{algorithm}
\usepackage[indLines=false]{algpseudocodex}
\DeclareMathOperator*{\Res}{Res}
\renewcommand\thesection{\arabic{section}}
\renewcommand\thesubsection{\thesection.\arabic{subsection}}
\renewcommand\thesubsubsection{\thesubsection.\arabic{subsubsection}}

\renewcommand\thesectiondis{\arabic{section}}
\renewcommand\thesubsectiondis{\thesectiondis.\arabic{subsection}}
\renewcommand\thesubsubsectiondis{\thesubsectiondis.\arabic{subsubsection}}
\renewcommand\thetable{\arabic{table}}
% correct bad hyphenation here
\hyphenation{op-tical net-works semi-conduc-tor}
\def\inputGnumericTable{}                                 %%

\renewcommand\algorithmicensure{\textbf{Input:}}
\newcommand{\algorithmautorefname}{Algorithm}

\lstset{
%language=C,
frame=single, 
breaklines=true,
columns=fullflexible,
literate=
{-}{$\rightarrow{}$}{1},
}
%\lstset{
%language=tex,
%frame=single, 
%breaklines=true
%}

\DeclareMathOperator*{\argmax}{arg\,max}
\DeclareMathOperator*{\argmin}{arg\,min}
\begin{document}
\newtheorem{theorem}{Theorem}[section]
\newtheorem{problem}{Problem}
\newtheorem{proposition}{Proposition}[section]
\newtheorem{lemma}{Lemma}[section]
\newtheorem{corollary}[theorem]{Corollary}
\newtheorem{example}{Example}[section]
\newtheorem{definition}[problem]{Definition}
\newcommand{\BEQA}{\begin{eqnarray}}
\newcommand{\EEQA}{\end{eqnarray}}
\newcommand{\define}{\stackrel{\triangle}{=}}
\bibliographystyle{IEEEtran}
\providecommand{\mbf}{\mathbf}
\providecommand{\pr}[1]{\ensuremath{\Pr\left(#1\right)}}
\providecommand{\qfunc}[1]{\ensuremath{Q\left(#1\right)}}
\providecommand{\sbrak}[1]{\ensuremath{{}\left[#1\right]}}
\providecommand{\lsbrak}[1]{\ensuremath{{}\left[#1\right.}}
\providecommand{\rsbrak}[1]{\ensuremath{{}\left.#1\right]}}
\providecommand{\brak}[1]{\ensuremath{\left(#1\right)}}
\providecommand{\lbrak}[1]{\ensuremath{\left(#1\right.}}
\providecommand{\rbrak}[1]{\ensuremath{\left.#1\right)}}
\providecommand{\cbrak}[1]{\ensuremath{\left\{#1\right\}}}
\providecommand{\lcbrak}[1]{\ensuremath{\left\{#1\right.}}
\providecommand{\rcbrak}[1]{\ensuremath{\left.#1\right\}}}
\theoremstyle{remark}
\newtheorem{rem}{Remark}
\newcommand{\sgn}{\mathop{\mathrm{sgn}}}
\providecommand{\abs}[1]{\left\vert#1\right\vert}
\providecommand{\res}[1]{\Res\displaylimits_{#1}} 
\providecommand{\norm}[1]{\left\lVert#1\right\rVert}
\providecommand{\mtx}[1]{\mathbf{#1}}
\providecommand{\mean}[1]{E\left[ #1 \right]}   
\providecommand{\fourier}{\overset{\mathcal{F}}{ \rightleftharpoons}}
\providecommand{\system}[1]{\overset{\mathcal{#1}}{ \longleftrightarrow}}
\newcommand{\solution}{\noindent \textbf{Solution: }}
\newcommand{\cosec}{\,\text{cosec}\,}
\providecommand{\dec}[2]{\ensuremath{\overset{#1}{\underset{#2}{\gtrless}}}}
\newcommand{\myvec}[1]{\ensuremath{\begin{pmatrix}#1\end{pmatrix}}}
\newcommand{\mydet}[1]{\ensuremath{\begin{vmatrix}#1\end{vmatrix}}}
\renewcommand{\vec}[1]{\boldsymbol{\mathbf{#1}}}
\def\putbox#1#2#3{\makebox[0in][l]{\makebox[#1][l]{}\raisebox{\baselineskip}[0in][0in]{\raisebox{#2}[0in][0in]{#3}}}}
     \def\rightbox#1{\makebox[0in][r]{#1}}
     \def\centbox#1{\makebox[0in]{#1}}
     \def\topbox#1{\raisebox{-\baselineskip}[0in][0in]{#1}}
     \def\midbox#1{\raisebox{-0.5\baselineskip}[0in][0in]{#1}}

\vspace{3cm}
\title{Beacon Tracking Using ESP32}
\author{
    \IEEEauthorblockN{Gautam Singh, G.V.V. Sharma} \\
    \IEEEauthorblockA{Indian Institute of Technology Hyderabad, India} \\
    Email: cs21btech11018@iith.ac.in, gadepall@ee.iith.ac.in
}
\maketitle
% \tableofcontents
\bigskip

\begin{abstract}
    This document is a report which demonstrates the use of machine learning in
    beacon tracking using an unmanned ground vehicle (UGV) and a WiFi-enabled
    microcontroller such as the ESP32.
\end{abstract}

\section{Introduction}

\section{Related Work}

\section{Working}
To estimate (radial) distance to beacon, we use its signal strength. For WiFi,
this is the \textbf{Received Signal Strength Indicator} (RSSI). The RSSI (in
dBm) at radial distance of \(r\) metres is given by
\begin{align}
    R\brak{r} = R\brak{1} - 10\log_{10}\brak{r}
    \label{eq:rssi}
\end{align}
where \(R\brak{1}\) is the RSSI at a distance of 1 metre from the beacon. The
beacon tracking problem can be formulated as the following optimization problem.
\begin{equation}
    \max_{r} R\brak{r} \text{ s.t. } r > 0.
    \label{eq:rssopt}
\end{equation}

The derivative and second derivative of \(R\brak{r}\) is given by
\begin{align}
    R^\prime\brak{r} &= -\frac{10}{\ln{10}}\frac{1}{r}, \label{eq:rssid} \\
    R^{\prime\prime}\brak{r} &= \frac{10}{\ln{10}}\frac{1}{r^2} > 0. \label{eq:rssidd}
\end{align}
Notice that for \(r > 0\), \(R^\prime\brak{r} < 0\), thus \(R\brak{r}\) is a
decreasing function of \(r\). This implies that the maximum RSSI is at \(r =
0\), as expected. Since \(R\brak{r}\) is a convex function of \(r\), we can use
gradient ascent to recursively find the point where the RSSI is maximum, which
would correspond to the location of the beacon. Using \eqref{eq:rssid}, the
gradient ascent update equation is given by
\begin{equation}
    r_{n+1} = r_n + \alpha R^\prime\brak{r_n} = r_n - \frac{10\alpha}{r_n\ln{10}}
    \label{eq:gradient-ascent-upd}
\end{equation}
where \(r_i\) is the radial distance at the \(i\)-th step and \(\alpha\) is the
step size. Since \(r_n > 0\), we can see that \(r_{n+1} < r_n\) for all \(n\).
In other words, the radial distance to the beacon is decreasing with each step.
This means that the UGV will converge towards the beacon. However, due to the
explosion of the gradient in \eqref{eq:rssid}, the update steps become larger as
\(r_n\) decreases, which could lead to overshooting the beacon. To prevent this,
the UGV uses a recursive algorithm to update its position using this principle
until it is close enough to the beacon based on the RSSI measurements it takes
at various points in the vicinity of its current position. The algorithm is
described in \autoref{alg:beacon}.

\begin{algorithm}[H]
    \caption{Beacon Tracking Algorithm}
    \label{alg:beacon}
    \begin{algorithmic}[1]
        \Ensure{RSSI threshold \(T\), number of steps \(N\)}
        \While{\textsc{GetRSSI()} \(< T\)}
            \State Take \(N\) steps in a straight line and measure the RSSI at 
            each step.
            \State Suppose the maximum RSSI is measured at step \(i\).
            \State Move to the position at step \(i\)
            \If{\(i = N\)}
                \State Move one step forward.
            \ElsIf{\(i = 0\)}
                \State Move one step backward.
            \Else
                \State Turn left.
            \EndIf
        \EndWhile
    \end{algorithmic}
\end{algorithm}

\section{Implementation}

\subsection{Assets}
\begin{enumerate}
    \item UGV chassis with DC motors
    \item ESP32 microcontroller with Type-B USB cable
    \item L293D Motor Driver IC
    \item Breadboard and Jumper Wires
    \item Android phone
    \item (Optional) USB 2.0/3.0 Hub
\end{enumerate}

\subsection{Procedure}
\begin{enumerate}
    \item Make the connections as per the wiring diagram in
    \autoref{fig:beacon}.
    \item Connect the ESP32 board to your Android Phone.
    \item Generate the firmware by entering the following commands.
        \begin{lstlisting}
$ cd codes
$ pio run
        \end{lstlisting}
    \item Go to ArduinoDroid and select
        \begin{lstlisting}
Actions - Upload - Upload Precompiled
        \end{lstlisting}
    and choose the firmware file at
        \begin{lstlisting}
codes/.pio/build/firmware.hex
        \end{lstlisting}
    \item Now put the phone at a reasonable distance from the UGV with no
    obstacles in the way and then turn on the hotspot. The UGV should travel
    towards the phone and stop near it.
\end{enumerate}

\begin{figure}[!ht]
    \centering
    \includegraphics[width=\columnwidth]{figs/beacon.png}
    \caption{Wiring Diagram for Beacon Tracking.}
    \label{fig:beacon}
\end{figure}

\section{Results}
The UGV eventually converges close to the beacon (here, the hotspot). However,
if there are a lot of nearby obstacles, the UGV may not converge close to the
location of the beacon. It may either get physically blocked by the beacon or
the signal interference may be too high.

\end{document}
